\documentclass[paper=a4paper,fontsize=9pt]{jlreq}
\usepackage{luatexja-fontspec}
\setmainfont{Harano Aji Mincho}
\setsansfont{Harano Aji Gothic}
\setmainjfont{Harano Aji Mincho}
\setsansjfont{Harano Aji Gothic}

\usepackage{amsmath,amssymb}
\usepackage{unicode-math}
\setmathfont{LatinModernMath-Regular}


\begin{document}

\title{証明:標本分散が母分散に確率収束}
\author{Sora Maekawa}
\maketitle

標本分散が母分散 \(\sigma^2\) に確率収束することを示す.
標本分散の定義は以下の通り:
\[
S^2 = \frac{1}{n-1} \sum_{i=1}^n (X_i - \bar{X})^2
\]
ここで, \(\bar{X}\) は標本平均で、次のように定義される:
\[
\bar{X} = \frac{1}{n} \sum_{i=1}^n X_i
\]

標本分散を展開し, 母分散 \(\sigma^2\) に関係づけるため次の変形を行う:
\[
X_i - \bar{X} = (X_i - \mu) - (\bar{X} - \mu),
\]
これを \( S^2 \) の定義に代入:
\[
S^2 = \frac{1}{n-1} \sum_{i=1}^n \left[ (X_i - \mu) - (\bar{X} - \mu) \right]^2 = \frac{1}{n-1} \sum_{i=1}^n (X_i - \mu)^2 - \frac{n}{n-1} (\bar{X} - \mu)^2
\]

大数の法則を適用し, 標本分散が母分散に収束することを示す.

(第1項):\(\frac{1}{n-1} \sum_{i=1}^n (X_i - \mu)^2\)
LLNと確率収束の性質より, i.i.d RVs \( X_1, X_2, \dots, X_n \) に対し:
\[
\frac{1}{n-1} \sum_{i=1}^n (X_i - \mu)^2 = \frac{n}{n-1} \times \frac{1}{n} \sum_{i=1}^n (X_i - \mu)^2 \xrightarrow{P} \mathbb{E}[(X - \mu)^2] = \sigma^2
\]
\( \lim_{n \rightarrow \infty}\frac{n}{n-1} = 1\) で収束することを利用した. したがって, この項は母分散 \(\sigma^2\) に確率収束する.

(第2項):\( \frac{n}{n-1}(\bar{X} - \mu)^2 \)
標本平均 \(\bar{X}\) も大数の法則により母平均 \(\mu\) に確率収束する:
\[
\bar{X} \xrightarrow{P} \mu
\]
このとき、\((\bar{X} - \mu)^2\) は確率収束の連続性(連続写像定理)により:
\[
(\bar{X} - \mu)^2 \xrightarrow{P} 0
\]
ここでも同様に\( \lim_{n \rightarrow \infty}\frac{n}{n-1} = 1\) で収束することを利用でき:
\[
\frac{n}{n-1}(\bar{X} - \mu)^2 \xrightarrow{P} 0
\]

上記の (1) と (2) を統合すると:
\[
S^2 = \frac{1}{n-1} \sum_{i=1}^n (X_i - \mu)^2 - \frac{n}{n-1} (\bar{X} - \mu)^2
\]
において, 第1項は \(\sigma^2\) に確率収束し, 第2項の影響は無視できる(確率収束で0)ため以下を得る:
\[
S^2 \xrightarrow{P} \sigma^2
\]

\end{document}